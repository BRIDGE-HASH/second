#说明我用的是jupyter进行代码运行这段代码的核心功能是:自动化获取特定品牌的新闻信息,并利用 GPT-4 模型进行情感分析和关键词提取。

核心逻辑分解
新闻数据抓取:

使用 SerpAPI 的 Google News 引擎,搜索与指定品牌(如“特斯拉”)相关的新闻。

通过设置参数(如搜索关键词、结果数量等),获取最新的新闻结果。

情感分析与关键词提取:

调用 OpenAI 的 GPT-4 模型,对每条新闻的描述进行分析。

提取出新闻的主题关键词,并判断其情感倾向(正面、负面或中性)。

结果输出:

将每条新闻的日期和分析结果打印输出,便于用户查看。


模块说明
SerpAPI:用于从 Google News 获取新闻数据。

OpenAI GPT-4:用于对新闻内容进行自然语言处理,提取关键词并分析情感。

环境变量:通过设置 SERPAPI_KEY 和 OPENAI_API_KEY,确保 API 调用的安全性和灵活性。

应用场景
品牌舆情监控:自动收集并分析品牌相关的新闻,了解公众情绪。

市场研究:获取竞争对手的最新动态,并进行情感分析。

内容策划:根据新闻热点和情感倾向,制定内容创作策略。





!pip install requests openai




# 通过 Jupyter 自带的 %env 魔法,确保变量对后续所有单元格生效
%env SERPAPI_KEY=在此填入_你的_真实_SerpApi_Key
%env OPENAI_API_KEY=在此填入_你的_OpenAI_Key




import os

key = os.getenv("SERPAPI_KEY")
print("SERPAPI_KEY repr:", repr(key))
print("Length       :", len(key) if key else None)

if not key:
    print("❌ SERPAPI_KEY 未设置,请返回单元格 2 重新设置!")
elif len(key) < 20 or " " in key:
    print("⚠️ SERPAPI_KEY 看起来不正确,请检查复制粘贴是否完整且无多余空格")  # 长度至少20位以上
else:
    print("✅ SERPAPI_KEY 格式正常,可进行下一步")






import requests, json, openai, os

# 先验证拉取新闻是否能成功
params = {
    "engine":  "google_news",
    "q":       "特斯拉",                   # 测试用热门关键词
    "num":     5,
    "api_key": os.getenv("SERPAPI_KEY")    # 一定要这里传入
}
r = requests.get("https://serpapi.com/search", params=params)
print("HTTP 状态码:", r.status_code)
print("接口返回示例:", r.json())            # 正确情况下应包含 "news_results" 字段:contentReference[oaicite:2]{index=2}

# 若上面返回200且有 news_results,再做简单分析示例
if r.status_code == 200 and "news_results" in r.json():
    openai.api_key = os.getenv("OPENAI_API_KEY")
    def analyze(text: str) -> dict:
        resp = openai.ChatCompletion.create(
            model="gpt-4o-mini",
            messages=[{"role":"user","content":f"提取关键词并判断情感:{text}"}],
            max_tokens=80,
            temperature=0
        )
        return json.loads(resp.choices[0].message.content)
    
    news = r.json()["news_results"]
    print(f"共抓取到 {len(news)} 条新闻,开始分析:")
    for item in news:
        desc = item.get("description","").strip()
        if desc:
            print("—", analyze(desc))






import requests, os

params = {
    "engine":  "google_news",
    "q":       "特斯拉",
    "num":     5,
    "api_key": os.getenv("SERPAPI_KEY")   # 必须在这里传入私钥
}

r = requests.get("https://serpapi.com/search", params=params)
print("HTTP 状态码:", r.status_code)
print("响应 JSON:", r.json())







#拉取新闻并调用 OpenAI 分析
import openai, json
openai.api_key = os.getenv("OPENAI_API_KEY")

def analyze(text: str) -> dict:
    resp = openai.ChatCompletion.create(
        model="gpt-4o-mini",
        messages=[{"role":"user","content":f"提取关键词并判断情感:{text}"}],
        max_tokens=80,
        temperature=0
    )
    return json.loads(resp.choices[0].message.content)

# 使用已验证的 fetch
news = r.json().get("news_results", [])
print(f"共获取到 {len(news)} 条新闻")
for item in news:
    date = item.get("date","")
    desc = item.get("description","").strip()
    if desc:
        print(f"{date} →", analyze(desc))
